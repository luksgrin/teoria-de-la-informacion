\section{Teoría de la Información}\label{teoruxeda-de-la-informaciuxf3n}

% \subsection{Índice}\label{uxedndice}

% \begin{enumerate}
% \def\labelenumi{\arabic{enumi}.}
% \tightlist
% \item
%   \hyperref[introducciuxf3n]{Introducción}
% \item
%   \href{./2-entropia/README.md}{Entropía}
% \item
%   \href{./3-estructura_lenguaje/README.md}{Estructura del lenguaje}
% \item
%   \href{./4-fuentes_informacion/README.md}{Fuentes de Información}
% \end{enumerate}

\subsection{Introducción}\label{introducciuxf3n-teoria-informacion}

La teoría de la información es una rama de las matemáticas que estudia
la cuantificación de la información. Fue propuesta por Claude Shannon en
1948 para estudiar la transmisión de mensajes en sistemas de
comunicación, y desde entonces se ha convertido en una disciplina
fundamental en la ingeniería de la comunicación y en la teoría de la
computación.

La teoría de la información se basa en la teoría de la probabilidad y en
la teoría de los sistemas de comunicación. Su objetivo es estudiar la
cantidad de información que se puede transmitir a través de un canal de
comunicación, y las limitaciones teóricas y prácticas que existen para
la transmisión de información.

\subsubsection{Esquema general de los sistemas de
comunicación}\label{esquema-general-de-los-sistemas-de-comunicaciuxf3n}

\begin{figure}[htbp!]
\centering
\includesvg{./img/shannon_communication_system.svg}
\end{figure}

En general, un sistema de comunicación consta de los siguientes
elementos:

\begin{itemize}
\tightlist
\item
  \textbf{Fuente de información}: es el origen de los mensajes que se
  van a transmitir. Puede ser un ser humano, un sensor, un ordenador,
  etc.
\item
  \textbf{Codificador}: es el encargado de transformar los mensajes de
  la fuente en una forma adecuada para su transmisión a través del canal
  de comunicación.
\item
  \textbf{Canal de comunicación}: es el medio físico a través del cual
  se transmiten los mensajes. Puede ser un cable, una fibra óptica, el
  aire, etc.
\item
  \textbf{Decodificador}: es el encargado de transformar los mensajes
  recibidos a través del canal en una forma adecuada para su
  interpretación por el destinatario.
\item
  \textbf{Destinatario}: es el receptor de los mensajes transmitidos por
  el sistema de comunicación.
\item
  \textbf{Ruido}: es cualquier perturbación que afecta a la transmisión
  de los mensajes a través del canal de comunicación.
\end{itemize}

Es debido a la presencia de ruido en el canal de comunicación que no es
posible transmitir información de forma perfecta. Por ese motivo, es
importante cuantificar la cantidad de información que se puede
transmitir a través de un canal de comunicación, y estudiar las
limitaciones teóricas y prácticas que existen para la transmisión de
información.
