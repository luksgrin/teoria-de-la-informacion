\documentclass{article}
\usepackage{svg}
\usepackage{amsmath}
\usepackage{amsfonts}
\usepackage{longtable}
\usepackage{booktabs}
\usepackage{hyperref}
\usepackage{mdframed}
\providecommand{\tightlist}{%
  \setlength{\itemsep}{0pt}\setlength{\parskip}{0pt}}
\usepackage{color}
\usepackage{fancyvrb}
\newcommand{\VerbBar}{|}
\newcommand{\VERB}{\Verb[commandchars=\\\{\}]}
\DefineVerbatimEnvironment{Highlighting}{Verbatim}{commandchars=\\\{\}}
% Add ',fontsize=\small' for more characters per line
\usepackage{framed}
\definecolor{shadecolor}{RGB}{248,248,248}
\newenvironment{Shaded}{\begin{snugshade}}{\end{snugshade}}
\newcommand{\KeywordTok}[1]{\textcolor[rgb]{0.13,0.29,0.53}{\textbf{#1}}}
\newcommand{\DataTypeTok}[1]{\textcolor[rgb]{0.13,0.29,0.53}{#1}}
\newcommand{\DecValTok}[1]{\textcolor[rgb]{0.00,0.00,0.81}{#1}}
\newcommand{\BaseNTok}[1]{\textcolor[rgb]{0.00,0.00,0.81}{#1}}
\newcommand{\FloatTok}[1]{\textcolor[rgb]{0.00,0.00,0.81}{#1}}
\newcommand{\ConstantTok}[1]{\textcolor[rgb]{0.00,0.00,0.00}{#1}}
\newcommand{\CharTok}[1]{\textcolor[rgb]{0.31,0.60,0.02}{#1}}
\newcommand{\SpecialCharTok}[1]{\textcolor[rgb]{0.00,0.00,0.00}{#1}}
\newcommand{\StringTok}[1]{\textcolor[rgb]{0.31,0.60,0.02}{#1}}
\newcommand{\VerbatimStringTok}[1]{\textcolor[rgb]{0.31,0.60,0.02}{#1}}
\newcommand{\SpecialStringTok}[1]{\textcolor[rgb]{0.31,0.60,0.02}{#1}}
\newcommand{\ImportTok}[1]{#1}
\newcommand{\CommentTok}[1]{\textcolor[rgb]{0.56,0.35,0.01}{\textit{#1}}}
\newcommand{\DocumentationTok}[1]{\textcolor[rgb]{0.56,0.35,0.01}{\textbf{\textit{#1}}}}
\newcommand{\AnnotationTok}[1]{\textcolor[rgb]{0.56,0.35,0.01}{\textbf{\textit{#1}}}}
\newcommand{\CommentVarTok}[1]{\textcolor[rgb]{0.56,0.35,0.01}{\textbf{\textit{#1}}}}
\newcommand{\OtherTok}[1]{\textcolor[rgb]{0.56,0.35,0.01}{#1}}
\newcommand{\FunctionTok}[1]{\textcolor[rgb]{0.00,0.00,0.00}{#1}}
\newcommand{\VariableTok}[1]{\textcolor[rgb]{0.00,0.00,0.00}{#1}}
\newcommand{\ControlFlowTok}[1]{\textcolor[rgb]{0.13,0.29,0.53}{\textbf{#1}}}
\newcommand{\OperatorTok}[1]{\textcolor[rgb]{0.81,0.36,0.00}{\textbf{#1}}}
\newcommand{\BuiltInTok}[1]{#1}
\newcommand{\ExtensionTok}[1]{#1}
\newcommand{\PreprocessorTok}[1]{\textcolor[rgb]{0.56,0.35,0.01}{\textit{#1}}}
\newcommand{\AttributeTok}[1]{\textcolor[rgb]{0.77,0.63,0.00}{#1}}
\newcommand{\RegionMarkerTok}[1]{#1}
\newcommand{\InformationTok}[1]{\textcolor[rgb]{0.56,0.35,0.01}{\textbf{\textit{#1}}}}
\newcommand{\WarningTok}[1]{\textcolor[rgb]{0.56,0.35,0.01}{\textbf{\textit{#1}}}}
\newcommand{\AlertTok}[1]{\textcolor[rgb]{0.94,0.16,0.16}{#1}}
\newcommand{\ErrorTok}[1]{\textcolor[rgb]{0.64,0.00,0.00}{\textbf{#1}}}
\newcommand{\NormalTok}[1]{#1}

\title{Teoría de la Información\\Ejercicios de Códigos}
\date{Curso 2023 - 2024}
\author{Lucas Goiriz Beltrán\\ Instituto de Biología Integrativa de Sistemas\\(I$_2$SysBio; UV-CSIC)\\Departamento de Matemática Aplicada\\Universitat Politècnica de València (UPV)}

\begin{document}
\maketitle

\textbf{
1. Sea la codificación bloque $f:\mathcal{A}^+\rightarrow\mathcal{B}^+$ en la que $\mathcal{A}=\left\{a,b,c\right\}$, $\mathcal{B}=\left\{0,1\right\}$ y $f(a)=0$, $f(b) = 01$, $f(c)=11$. ¿Es $f$ unívocamente decodificable?
}

\vspace{1cm}

\textbf{
2. Sea la codificación bloque $f:\mathcal{A}^+\rightarrow\mathcal{B}^+$ en la que $\mathcal{A}=\left\{a,b,c,d\right\}$, $\mathcal{B}=\left\{a,b\right\}$ y $f(a)=ab$, $f(b) = aaab$, $f(c)=aba$, $f(d)=aab$. ¿Es $f$ unívocamente decodificable?
}

\vspace{1cm}

\textbf{
3. Sea la codificación bloque $f:\mathcal{A}^+\rightarrow\mathcal{B}^+$ en la que $\mathcal{A}=\left\{a,b,c,d\right\}$, $\mathcal{B}=\left\{a,b\right\}$ y $f(a)=aba$, $f(b) = a$, $f(c)=bab$, $f(d)=bb$. ¿Es $f$ unívocamente decodificable?
}

\vspace{1cm}

\textbf{
4. Sea la codificación bloque $f:\mathcal{A}^+\rightarrow\mathcal{B}^+$ en la que $\mathcal{A}=\left\{a,b,c,d\right\}$, $\mathcal{B}=\left\{a,b\right\}$ y $f(a)=a$, $f(b) = abb$, $f(c)=aba$, $f(d)=bab$. ¿Es $f$ unívocamente decodificable?
}

\vspace{1cm}

\textbf{
5. Sea la codificación bloque $f:\mathcal{A}^+\rightarrow\mathcal{B}^+$ en la que $\mathcal{A}=\left\{a,b,c,d\right\}$, $\mathcal{B}=\left\{a,b\right\}$ y $f(a)=aba$, $f(b) = ab$, $f(c)=abb$, $f(d)=bbcb$. ¿Es $f$ unívocamente decodificable?
}

\pagebreak

\textbf{
6. Sea la codificación bloque $f:\mathcal{A}^+\rightarrow\mathcal{B}^+$ en la que $\mathcal{A}=\left\{a,b,c,d,e\right\}$, $\mathcal{B}=\left\{0,1\right\}$ y $\left|f(a)\right|=4$, $\left|f(b)\right|=4$, $\left|f(c)\right|=1$, $\left|f(d)\right|=2$, $\left|f(e)\right|=3$.
}

\begin{itemize}
    \item\textbf{Demuestra que cumple la igualdad de Kraft}.
    \item\textbf{Particulariza $f$ para que $f\left(\mathcal{A}\right)$ sea un código completo alfabéticamente ajustado a la ordenación generada por $0<1$}.
\end{itemize}

\vspace{1cm}

\textbf{
7. Sea la codificación bloque $f:\mathcal{A}^+\rightarrow\mathcal{B}^+$ en la que $\mathcal{A}=\left\{a_1,a_2,a_3,a_4,a_5,a_6,a_7\right\}$, $\mathcal{B}=\left\{a,b,c\right\}$ con $\left|f(a_1)\right|=3$, $\left|f(a_2)\right|=2$, $\left|f(a_3)\right|=2$, $\left|f(a_4)\right|=3$, $\left|f(a_5)\right|=1$, $\left|f(a_6)\right|=3$, $\left|f(a_7)\right|=1$. ¿Cumple $f$ con la desigualdad de Kraft? Define $f$, si es posible, de modo que ea una codificación bloque instantánea alfabéticamente ordenada. ¿Es completa?
}

\vspace{1cm}

\textbf{
8. Sea una fuente de memoria nula $FI=\left(\mathcal{A},p\right)$ con $\mathcal{A}=\left\{a_1,a_2,a_3,a_4\right\}$ y probabilidades $p(a_1)=0.4$, $p(a_2)=0.3$, $p(a_3)=0.2$, $p(a_4)=0.1$. Obtén una codificación bloque instantánea $f$ con el alfabéto código $\left\{0,1\right\}$ para la fuente de información.
}

\vspace{1cm}

\textbf{
9. Sea una fuente de memoria nula $FI=\left(\mathcal{A},p\right)$ con $\mathcal{A}=\left\{a_1,a_2,a_3,a_4,a_5,a_6\right\}$ y probabilidades $p(a_1)=0.  $, $p(a_2)=0.   $, $p(a_3)=0.2$, $p(a_4)=0.2$, $p(a_5)=0.3$ y $p(a_6)=0.1$. Obtén una codificación bloque instantánea $f$ con el alfabéto código $\left\{0,1\right\}$ para la fuente de información.
}

\end{document}