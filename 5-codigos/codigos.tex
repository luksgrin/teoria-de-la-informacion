\subsection{Códigos}\label{cuxf3digos}

\subsubsection{Introducción}\label{introducciuxf3n-codigos}

Como hemos visto al principio del temario, existe un componente en el
esquema de transmisión de información llamado \textbf{codificador}, el
cual se encarga de transformar los mensajes de la fuente (\texttt{MF})
en una forma adecuada para su transmisión a través del canal de
comunicación y recepción (mensajes codificados, \texttt{MC}), donde se
lleva a cabo el proceso correspondiente de decodificación para la
recuperación del mensaje fuente original \texttt{MF}.

En una primera aproximación simplificada a este escenario, definimos el
proceso de codificación mediante una función \(f\) tal que

\[
f:\text{\texttt{MF}}\rightarrow\mathcal{P}(\text{\texttt{MC}})
\]

donde \(\forall m\in\text{\texttt{MC}}: \left|f(m)\right|<\infty\) (es
decir que los resultados de la codificación del mensaje tienen una
longitud finita). Nótese que \(\mathcal{P}(C)\) significa el conjunto
potencia de \(C\).

Por lo tanto, en este contexto de transmisión de información, cuando se
emite un mensaje aletorio \(m\), el receptor recibe un mensaje
codificado \(m'\in\mathcal{P}(C)\). Entonces, existe en el receptor una
función \(f^\star\) de decodificación:

\[
f^\star:\text{\texttt{MC}}\rightarrow\text{\texttt{MF}}
\]

tal que

\[
\forall m\in \text{\texttt{MF}}, \quad\exists m': f^\star\left(m'\right) = m
\]

Por ahora asumiremos que
\(\forall m\in \text{\texttt{MF}}, \left|f(m)\right| = 1\) (pero veremos
que nos podremos extender a más casos).

\subsubsection{Codificaciones y
códigos}\label{codificaciones-y-cuxf3digos}

Llevando la introducción anterior a un caso más particular (y familiar);
sea \(\mathcal{A}\) un alfabeto (correspondiente a la fuente de
información de memoria nula \((\mathcal{A},P)\), de la que se originan
los mensajes fuente \(\mathcal{A^+}\)) y sea \(\mathcal{C}\) el conjunto
de los mensajes codificados \texttt{MC} (también conocido como el
conjunto de códigos), entonces una codificación \(f\) es una aplicación
\textbf{biyectiva}:

\[
f:\mathcal{A}^+\rightarrow\mathcal{C}
\]

Para que esta codificación sea admisible para un proceso eficiente de
codificación-decodificación, deberá tener unas características
determinadas que se irán exponiendo más adelante.

Sea \(\mathcal{B}\) un alfabeto (código), entonces
\(\mathcal{C}\subseteq\mathcal{B}\). Entonces una codificación admisible
es una aplicación inyectiva

\[
f:\mathcal{A}^+\rightarrow\mathcal{B}^+
\]

en la que tenemos que \(f\left(\mathcal{A}^+\right) = \mathcal{C}\).
Equivalentemente, si tomamos \(f(\lambda) = \lambda\) (y también
\(f^-1(\lambda) = \left\{\lambda\right\}\)), entonces podemos extender
esta aplicación inyectiva a

\[
f:\mathcal{A}^*\rightarrow\mathcal{B}^*
\]

En este escenario \(\mathcal{A}\) recibe el nombre de ``\emph{alfabeto
fuente}'' y \(\mathcal{B}\) ``\emph{alfabeto código}''.

\subsubsection{Codificación Bloque}\label{codificaciuxf3n-bloque}

Una codificación bloque \(f\) es una aplicación que le asigna a cada uno
de los símbolos \(a\in\mathcal{A}\) una palabra de \(\mathcal{B}^+\), y
se comporta como un homomorfismo, lo cual quiere decir que

\[
\forall x\in\mathcal{A}^+,\forall a\in\mathcal{A}: f(ax) = f(a)f(x)
\]

y de forma equivalente

\[
\forall x\in\mathcal{A}^+,\forall a\in\mathcal{A}: f(xa) = f(x)f(a)
\]

y de forma general

\[
\forall x,y\in\mathcal{A}^+: f(xy) = f(x)f(y)
\]

Por tanto

\[
\forall x\in\mathcal{A}^+: f(x) = f(x[1])\dots f(x[|x|])
\]

\emph{(todo esto es equivalente, y lo hemos visto anteriormente en la
definición de homomorfismo en el apartado de estructura del lenguage)}

El conjunto
\(f(\mathcal{A})=\left\{f(a),\quad\forall\mathcal{A}\right\}\) recibe el
nombre de ``\emph{código bloque}'' (o simplemente ``\emph{código}'').

\paragraph{Ejemplo 1}\label{ejemplo-1-codigos}

Consideremos los siguientes alfabetos fuente (\(\mathcal{A}\)) y código
(\(\mathcal{B}\)):


\begin{align*}
\mathcal{A} &=\left\{0,1,2,3,4,5,6,7,8,9\right\}\\
\mathcal{B} &=\left\{0,1\right\}
\end{align*}


y además consideremos la codificación de bloque definida por \(f\):

\[
f(a) = \left(\left\lfloor\frac{a_1}{2^0}\right\rfloor\right)\left(\left\lfloor\frac{a_2}{2^1}\right\rfloor\right)\left(\left\lfloor\frac{a_3}{2^2}\right\rfloor\right)\left(\left\lfloor\frac{a_4}{2^3}\right\rfloor\right)
\]

claramente tenemos que

\[
f(\mathcal{A}) = \left\{0000, 0001, 0010, 0011, 0100, 0101, 0110, 0111, 1000, 1001\right\}
\]

Se puede ver que \(f(\mathcal{A})\subset\mathcal{B}^4\) (porque hay
palabras en \(\mathcal{B}^4\) que no aparecen en \(f(\mathcal{A})\),
como por ejemplo \(1111\)). \(f\) es una codificación de bloque ya que
le asigna a cada caracter del alfabeto fuente, una palabra formada por
el alfabeto código.

\paragraph{Ejemplo 2: Morse}\label{ejemplo-2-morse}

En el morse, el alfabeto código consta de los elementos
\(\left\{\cdot,-,\text{\texttt{pausa}}\right\}\):

\begin{longtable}[]{@{}ll@{}}
\toprule\noalign{}
Alfabeto fuente & Código \\
\midrule\noalign{}
\endhead
\bottomrule\noalign{}
\endlastfoot
A & .- \\
B & -\ldots{} \\
C & -.-. \\
D & -.. \\
E & . \\
F & ..-. \\
G & --. \\
H & \ldots. \\
I & .. \\
J & .--- \\
K & -.- \\
L & .-.. \\
M & -- \\
N & -. \\
P & .--. \\
Q & --- \\
S & \ldots{} \\
T & - \\
U & ..- \\
V & \ldots- \\
W & .-- \\
X & -..- \\
Y & -.-- \\
Z & --.. \\
1 & .---- \\
2 & ..--- \\
3 & \ldots-- \\
4 & \ldots.- \\
5 & \ldots.. \\
6 & -\ldots. \\
7 & --\ldots{} \\
8 & ---.. \\
9 & ----. \\
0 & ---- \\
\end{longtable}

Al codificar un mensaje en morse, después del código asociado a cada
simbolo ha de introducirse una pausa de separación.

\paragraph{Ejemplo 3: ASCII}\label{ejemplo-3-ascii}

El ASCII (\emph{American Standard Code for Information Interchange}) es
una codificación bloque de logitud constante (no como el morse, que era
de longitud variable) con alfabeto código
\(\mathcal{B}=\left\{0,1\right\}\) y
\(f(\mathcal{A})=\left\{0,1\right\}^7\).

\paragraph{Ejemplo 4: ASCII Extendido}\label{ejemplo-4-ascii-extendido}

El ASCII extendido es diferencia del ASCII por
\(f(\mathcal{A})=\left\{0,1\right\}^8\) (la longitud de las palabras es
mayor).

\paragraph{Unicode}\label{unicode}

Unicode es un alfabeto universal junto con la codificación de sus
símbolos (más de un millón) mediante un código bloque de longitud fija e
incorpora todos los símbolos del ASCII extendido. A partir de Unicode se
encuentran, entre otras, las codificaciones bloques de longitud variable
UTF (Unicode Transformation Format): UTF-8 (la más común), UTF-16 y
UTF-32.

\subsubsection{Codificación Bloque Unívocamente
Decodificable}\label{codificaciuxf3n-bloque-unuxedvocamente-decodificable}

Una codificación de bloque \(f\) se llama \textbf{no singular} siempre
que

\[
\forall a,b\in\mathcal{A}: a\neq b\Rightarrow f(a)\neq f(b)
\]

\paragraph{Ejemplo 1}\label{ejemplo-1-1-codigos}

Sea \(\mathcal{A} = \left\{a_1,a_2,a_3,a_4\right\}\), \(\mathcal{B} =
\left\{0,1\right\}\) y \(f\) definida por:

\[
f = \begin{cases}
f(a_1) &= 0\\
f(a_2) &= 11\\
f(a_3) &= 00\\
f(a_4) &= 01\\
\end{cases}
\]

Es sencilo ver que \(f\) es no singular. Sin embargo, si nos extendemos
a palabras, esto ya no se cumple ya que ciertas palabras de
\(\mathcal{A}^+\) no son unívocamente decodificables. Por ejemplo:

\begin{itemize}
\tightlist
\item
  \(a_1a_1a_1\neq a_1a_3\neq a_3a_1\) pero
  \(000 = f(a_1a_1a_1 ) = f(a_1a_3) = f(a_3a_1)\)
\item
  \(a_1a_1a_2\neq a_3a_2\) pero \(0011 = f(a_1a_1a_2) = f(a_3a_2)\)
\end{itemize}

\begin{center}\rule{0.5\linewidth}{0.5pt}\end{center}

La extensión de orden \(n\geq 1\)de una codificación bloque

\[
f:\mathcal{A}^+\rightarrow\mathcal{B}^+
\]

se define como

\[
f^{(n)}:\left(\mathcal{A}^n\right)^+\rightarrow\mathcal{B}^+
\]

tomando como símbolos las palabras de \(\mathcal{A}^n\) de forma que

\[
\forall x\in\mathcal{A}^n: f^{(n)}(x)=f(x)
\]

La extensión de orden \(n\) de una codificación bloque también lo es
tomando, para cada caso, como alfabeto fuente el correspondiente
\(\mathcal{A}\).

Una codificación bloque \(f\) se dice unívocamente decodificable sí y
sólo si su extensión de orden \(n\) es \textbf{no singular} para cada
\(n\geq 1\).

Surge una propiedad importante: una codificación bloque es unívocamente
decodificable si y sólo si es inyectiva.

Diremos que un código bloque es unívocamente decodificable si proviene
de una codificación bloque unívocamente decodificable, lo cual nos
permite definir \(f^{-1}:\mathcal{B}^+\rightarrow\mathcal{A}^+\).

\textbf{Propiedad de factorización única:}

\begin{enumerate}
\def\labelenumi{\arabic{enumi}.}
\tightlist
\item
  Si \(f:\mathcal{A}^+\rightarrow\mathcal{B}^+\) es una codificación
  bloque unívocamente decodificable, entonces se tiene que
  \(\forall x_1,\dots,x_n,y_1,\dots,y_m\in f(\mathcal{A})\):
\end{enumerate}

\[
x_1\dots x_n=y_1\dots y_m \Rightarrow (n = m)\wedge\left(x_i=y_i,\quad i = 1,\dots,n\right)
\]

\begin{enumerate}
\def\labelenumi{\arabic{enumi}.}
\setcounter{enumi}{1}
\tightlist
\item
  \(f:\mathcal{A}^+\rightarrow\mathcal{B}^+\) (que es una codificación
  bloque) es unívocamente decodificable si y solamente si cumple la
  propiedad mencionada anteriormente, y además es no singular.
\end{enumerate}

La cuestión de si una codificación bloque \(f\) es unívocamente
decodificable o no es algorítmicamente decidible (\emph{es decir, que
hay un algoritmo para averiguar si es unívocamente decodificable o no}).

El siguiente algoritmo devuelve \texttt{true} cuando \(f\) es
unívocamente decodificable y \texttt{false} en caso contrario:


\begin{align*}
&\text{if }\left(\exists a,b\in\mathcal{A}: (a\neq b)\wedge(f(a)=f(b))\right)\text{then:}\\
&\quad\quad\text{return \texttt{false}}\\
&A=\left\{u: \left(\exists x,y\in f(\mathcal{A}):(x\neq y)\wedge (xu=y)\right)\right\}\\
&\text{if}\left(A\cap f(A)\neq\emptyset\right)\text{ then:}\\
&\quad\quad\text{return \texttt{false}}\\
&A'=\emptyset\\
&\text{while }A'\neq\emptyset:\\
&\quad\quad A'=A\cup A'\\
&\quad\quad B = \left\{u:\left(\left(\exists x\in f(\mathcal{A}): xu\in A\right)\vee\left(\exists x\in A: xu\in f(\mathcal{A})\right)\right)\right\}\\
&\quad\quad A = B - A'\\
&\quad\quad\text{if }\left(A\cap f(\mathcal{A})\neq\emptyset\right)\text{ then:}\\
&\quad\quad\quad\quad\text{return \texttt{false}}\\
&\text{return \texttt{true}}\\
\end{align*}


Sea \(f:\mathcal{A}^+\rightarrow\mathcal{B}^+\) una codificación bloque
no singular. Si todas las palabras del código bloque tienen exactamente
la misma longitud, entonces el código es unívocamente decodificable.

\emph{Piensa en el Ejemplo 1 de esta sección. ¿De dónde salían los
problemas a la hora de decodificar? Surgían del hecho de que no todas
lass palabras del código bloque tenían la misma longitud}

\subsubsection{Desigualdad de McMillan}\label{desigualdad-de-mcmillan}

Si \(f:\mathcal{A}^+\rightarrow\mathcal{B}^+\) es una codificación
bloque unívocamente decodificable y \(\left|\mathcal{B}\right|=k\),
entonces

\[
\sum_{x\in f(\mathcal{A})} k^{-\left|x\right|}\leq 1
\]

\paragraph{Ejemplo 1}\label{ejemplo-1-2}

Recuperando el ejemplo anterior de los siguientes alfabetos fuente
(\(\mathcal{A}\)) y código (\(\mathcal{B}\)):


\begin{align*}
\mathcal{A} &=\left\{0,1,2,3,4,5,6,7,8,9\right\}\\
\mathcal{B} &=\left\{0,1\right\}
\end{align*}


con

\[
f(\mathcal{A}) = \left\{0000, 0001, 0010, 0011, 0100, 0101, 0110, 0111, 1000, 1001\right\}
\]

La desigualdad de McMillan debería de cumplirse ya que es una
codificación bloque unívocamente decodificable:

\[
\sum_{x\in f(\mathcal{A})} k^{-\left|x\right|} = \sum_{x\in f(\mathcal{A})} 2^{-4} = \frac{10}{16} < 1
\]

Vemos rápidamente que se cumple.

\emph{¿Qué condición debería cumplirse para tener la igualdad estricta?}

Imaginémonos ahora que tenemos


\begin{align*}
\mathcal{A} &=\left\{0,1,2,3,4,5,6,7,8,9,a,b,c,d,e,f\right\}\\
\mathcal{B} &=\left\{0,1\right\}
\end{align*}


con

\[
f(\mathcal{A}) = \left\{0000, 0001, 0010, 0011, 0100, 0101, 0110, 0111, 1000, 1001,1010,1011,1100,1101, 1110,1111\right\}
\]

\emph{(¿no resulta esto familiar? Es la representación en binario del
hexadecimal)}

Para este caso:

\[
\sum_{x\in f(\mathcal{A})} k^{-\left|x\right|} = \sum_{x\in f(\mathcal{A})} 2^{-4} = \frac{16}{16} = 1
\]

Claramente cuando \(f(\mathcal{A})\subset\mathcal{B}^+\), entonces la
desigualdad es estricta. Pero cuando \(f(\mathcal{A})=\mathcal{B}^+\),
entonces tenemos la igualdad.

\subsubsection{Códigos instantáneos}\label{cuxf3digos-instantuxe1neos}

Empecemos observando un ejemplo. Consideremos la codificación
unívocamente decodificable \(f(a)=0\) y \(f(b)=01\). Para decodificar
\(x=01\), tendríamos que conocer de la existencia del \(1\), ya que
únicamente con el \(0\) tenemos una ambigüedad y no sabemos si se trata
de \(a\) o de \(b\). Al leer el \(1\), resolvemos esto y podemos
decodificar \(x\) como \(b\). Es decir, hasta no leer la palabra entera,
no hemos podido saber cómo decodificarla.

Esta situación empeora cuanto más larga es la palabra, y puede llevar a
que la decodificación sea impracticable.

Consideremos ahora la codificación unívocamente decodificable
\(f(a)=0\), \(f(b)=01\) y \(f(c)=11\). Sea \(x=01^n\) con \(n\geq 1\).
Tenemos que

\[
f^-1(x) = \begin{cases}
ac^m&\quad\text{si }n\text{ es par, con }m=\frac{n}{2}\\
bc^m&\quad\text{si }n\text{ es impar, con }m=\frac{n-1}{2}
\end{cases}
\]

Veamos una serie de ejemplos:

\begin{itemize}
\tightlist
\item
  \(f^-1(x)=f^-1(01^6)=f^-1(0111111)=accc\)
\item
  \(f^-1(x)=f^-1(01^{11})=f^-1(011111111111)=bccccc\)
\end{itemize}

Para decodificar, es esencial conocer cuantos \(1\) tenemos en la
palabra \(x\). Esto no lo podemos averiguar si no leemos toda la palabra
primero (independientemente de lo larga que sea esta), antes de
decodificar.

Se dice que un código unívocamente decodificable es \textbf{instantáneo}
cuando es posible decodificar cada símbolo de \(\mathcal{A}\) de cada
mensaje sin necesitar más símbolos de \(\mathcal{B}\) de los
estricamente necesarios. Es decir, si

\[
h:\mathcal{A}^*\rightarrow\mathcal{B}^*
\]

es una codificación bloque unívocamente decodificable, entonces es
instantánea siempre que

\[
\forall x\in\mathcal{A}^*, \exists u,v\in\mathcal{A}^*,\exists a\in\mathcal{A}: x=uav
\]

tal que

\[
y=h(x)=h(u)h(a)h(v)
\]

Entonces para decodificar \(y\) a \(x\), para decodificar el segmento
\(h(a)\) a \(a\) de forma inmediata, independientemente del sufijo
\(h(v)\), solamente hay que procesar el segmento de \(y\) asociado a
\(h(a)\).

Entonces para descodificar el segmento
\(y\left[\left|h(u)\right|+1:\left|h(u)\right|+n\right]\) con, en este
caso, \(n = \left|h(a)\right|\):


\begin{align*}
y\left[\left|h(u)\right|+1:\left|h(u)\right|+n\right] &= y\left[\left|h(u)\right|+1:\left|h(u)\right|+\left|h(a)\right|\right]\\
&= y\left[\left|h(u)\right|+1:\left|h(ua)\right|\right]
\end{align*}


es suficiente, en cualquier caso, para la correcta decodificación. Eso
quiere decir que

\[
\forall a,b\in\mathcal{A}:h(a)\text{ es prefijo de }h(b)\Rightarrow a = b
\]

Una vez más, diremos que un código bloque es instantáneo cuando proviene
de una codificación bloque instantánea. También podemos llamarlo código
bloque prefijo o simplemente código prefijo.

\begin{center}\rule{0.5\linewidth}{0.5pt}\end{center}

Sea \(f:\mathcal{A}^*\rightarrow\mathcal{B}^*\) una codificación de
bloque. Diremos que \(f\) es \textbf{estable} siempre que:

\begin{enumerate}
\def\labelenumi{\arabic{enumi}.}
\tightlist
\item
  Sea unívocamente decodificable.
\item
  \(\forall x\in\mathcal{A}^*,f(x)=u\) se cumple que
  \(\forall v\in\mathcal{B}^*\), si
  \(\exists y\in\mathcal{A}^*,f(y)=uv\Rightarrow y=xz\) para algún
  \(z\in\mathcal{A}^*\)
\end{enumerate}

Tenemos entonces la siguiente propiedad: sea
\(f:\mathcal{A}^*\rightarrow\mathcal{B}^*\) una codificación bloque.
Entonces, \textbf{\(f\) es estable si y sólo si es instantánea}.

\begin{center}\rule{0.5\linewidth}{0.5pt}\end{center}

\subsubsection{Desigualdad de Kraft}\label{desigualdad-de-kraft}

Veamos cómo estad características cualitativas se traducen
cuantitativamente. Sea la codificación bloque:

\[
f:\mathcal{A}^*\rightarrow\mathcal{B}^*
\]

con \(\mathcal{A}=\left\{a_1,\dots,a_n\right\}\),
\(\mathcal{B}=\left\{b_1,\dots,b_k\right\}\) y
\(f\left(\mathcal{A}\right)=\left\{x_1,\dots,x_n\right\}\), con
\(l_i = \left|x_i\right|\). La desigualdad de Kraft nos da la condición
\textbf{necesaria} y \textbf{suficiente} para que exista un código
bloque isntantáneo con palabras de su código bloque de longitud
\(l_1,\dots,l_n\) sobre un alfabeto con \(k\) símbolos:

\[
\sum_{1\leq i\leq n}k^{l_i}\leq 1
\]

Cuando esto es una igualdad estricta, se conoce como la igualdad de
Kraft.

Como consecuencia inmediata, a partir de la desigualdad de McMillan, se
tiene que para cada código unívocamente decodificable se tiene un código
instantáneo con las mismas longitudes de las palabras de su código
bloque.

\subsubsection{Códigos Completos}\label{cuxf3digos-completos}

Sea la codificación de bloque:

\[
f:\mathcal{A}^*\rightarrow\mathcal{B}^*
\]

diremos que es completa si:

\begin{enumerate}
\def\labelenumi{\arabic{enumi}.}
\tightlist
\item
  Es instantánea.
\item
  Cumple que
  \(\forall x\in\mathcal{B}^*,\exists a\in\mathcal{A}^*\Rightarrow\left(\left(h(a)\text{ es prefijo de }x\right)\vee\left(x\text{ es prefijo de }h(a)\right)\right)\).
\end{enumerate}

Tenemos entonces la siguiente propiedad: si
\(\left|\mathcal{A}\right|\geq 2\) y \(\left|\mathcal{B}\right|=2\),
entonces sólo existe una codificación de bloque completa

\[
f':\mathcal{A}^*\rightarrow\mathcal{B}^*
\]

tal que
\(\forall a\in\mathcal{A}:\left|f'(a)\right|\leq\left|f(a)\right|\).

Nuevamente, un código bloque es completo si proviene de una codificación
de bloque completa.

\subsection{Codificaciones de Fuentes de
Información}\label{codificaciones-de-fuentes-de-informaciuxf3n}

\subsubsection{Longitud media de un
código}\label{longitud-media-de-un-cuxf3digo}

Para un \(\mathcal{A}^*\) y \(\mathcal{B}^*\), es posible definir más de
una codificación instantánea o unívocamente decodificable. Esto requiere
entonces que intentemos elegir las más eficientes con el objetivo de
tener una transmisión de información óptima.

Un criterio natural de selección (aún cuando no es el único posible) es
el de la mínima longitud media.

Sea un código bloque que asocia los símbolos de una \textbf{fuente de
información de memoria nula} \(FI = \left(\mathcal{A},P\right)\) donde
\(\mathcal{A}=\left\{a_1,\dots,a_n\right\}\) y
\(\left|\mathcal{B}\right|=k\) mediante la codificación
\(f:\mathcal{A}^*\rightarrow\mathcal{B}^*\) con las palabras
\(f\left(a_i\right)=x_i\), con \(l_i=\left|x_i\right|,i=1,\dots,n\).
Definimos la lonfitud media como la \textbf{esperanza matemática de la
longitud de los códigos bloque}:

\[
\mathcal{L}_f=\mathbb{E}\left[l\right]=\mathbb{E}\left[\left|x\right|\right]=\mathbb{E}\left[\left|f(a)\right|\right]=\sum_{i=1}^n \left|f(a_i)\right|p(a_i)
\]

\subsubsection{Códigos óptimos}\label{cuxf3digos-uxf3ptimos}

Sea un código bloque unívocamente decodificable que asocia los símbolos
de la fuente de memoria nula \(FI = \left(\mathcal{A},P\right)\) con
palabras formadas por un alfabeto \(\left|\mathcal{B}\right|=k\).
Diremos que es \textbf{compacto} u \textbf{óptimo} con respecto a la
fuente de información si su longitud media es menor o igual que la
longitud media de cada uno de los códigos bloque unívocamente
decodificables que pueden definirse entre la fuente y alfabetos códigos
con \(k\) símbolos.

Surge de aquí la observación de que, puesto que esto incide únicamente
sobre las logitudes, la búsqueda puede reducirse por la desigualdad de
McMillan a códigos instantáneos.

Además, si se tiene que \(\left|\mathcal{A}\right|\geq 2\) y
\(\left|\mathcal{B}\right| = 2\), si un código bloque es óptimo,
entonces es completo.

También se tiene la propiedad de si \(f\) es un código óptimo, entonces

\[
\forall a,b\in\mathcal{A}:p(a)<p(b)\Rightarrow \left|f(a)\right|\geq\left|f(b)\right|
\]

\subsubsection{\texorpdfstring{Primer Teorema de Shannon \emph{(teorema
de la codificación sin
ruido)}}{Primer Teorema de Shannon (teorema de la codificación sin ruido)}}\label{primer-teorema-de-shannon-teorema-de-la-codificaciuxf3n-sin-ruido}

Consideremos una fuente de memoria nula \(FI\) cuyos símbolos
\(a_1,\dots,a_n\) con probabilidades \(p_1,\dots,p_n\) se codifican cada
uno en una palabra de longitud \(l_i\) en un alfabeto con \(k\) símbolos
mediante la codificación \(f\). Entonces se tiene que
\(H_k\left(FI\right)\leq\mathcal{L}_f\).

Si suponemos que nos encontramos en el caso de la igualdad, es decir,
\(H_k\left(FI\right)=\mathcal{L}_f\):


\begin{align*}
H_k\left(FI\right)&=\mathcal{L}_f\\
\sum_{i=1}^np_i\log_k\left(\frac{1}{p_i}\right)&=\sum_{i=1}^np_il_i
\end{align*}


inferimos que si tomamos longitudes de código
\(l_i=\left|x_i\right|=\left|f(a_i)\right|\), tendremos que los códigos
obtenidos para cada \(a_i\) serán instantáneos, completos y óptimos con
una longitud media que coincide con la entropía en base \(k\) de la
fuente de información (asumiendo claramente que
\(\log_k\left(\frac{1}{p_i}\right)\) es un número entero
\(\forall 1\leq i\leq n\)).

Veamos un ejemplo: supón que tenemos
\(FI=\left(\left\{a_1,a_2,a_3\right\},\left\{\frac{1}{2},\frac{1}{4},\frac{1}{4}\right\}\right)\),
entonces la codificación de bloque:

\[
h:\left\{a_1,a_2,a_3\right\}^*\rightarrow\left\{0,1\right\}^*
\]

definida tomando las longitudes de los códigos como
\(\log\left(\frac{1}{p_i}\right)\):

\[
h=\begin{cases}
h\left(a_1\right) &= 1\\
h\left(a_2\right) &= 00\\
h\left(a_3\right) &= 01
\end{cases}
\]

es instantánea, completa y óptima con
\(\mathcal{L}_h=H\left(FI\right)\).

Pero, ¿qué sucede si resulta que \(\log_k\left(\frac{1}{p_i}\right)\) no
resultan ser números enteros? Parece intuitivo y apropiado en este
contexto redondear el valor obtenido hacia arriba, para elegir \(l_i\)
(aunque claramente en este caso, el código obtenido no tiene por qué ser
optimo). Entonces si
\(l_i=\left\lceil\log_k\left(\frac{1}{p_i}\right)\right\rceil\), por
definición se tiene que

\[
\log_k\left(\frac{1}{p_i}\right)\leq l_i\leq\log_k\left(\frac{1}{p_i}\right) + 1
\]

Se puede demostrar que se cumple la desigualdad de Kraft. Tomemos
\(\log_k\left(\frac{1}{p_i}\right)\leq l_i\):


\begin{align*}
\log_k\left(\frac{1}{p_i}\right)&\leq l_i\\
k^{\log_k\left(\frac{1}{p_i}\right)}&\leq k^{l_i}\\
\frac{1}{p_i}&\leq k^{l_i}\\
\frac{1}{k^{l_i}}&\leq p_i\\
k^{-l_i}&\leq p_i,\quad\forall 1\leq i\leq n\\
\sum_{i=1}^nk^{-l_i}&\leq\sum_{i=1}^n p_i\\
\sum_{i=1}^nk^{-l_i}&\leq 1
\end{align*}


En consecuencia, pueden asociárseles códigos bloque instantáneos.

Volviendo a la siguiente desigualdad:

\[
\log_k\left(\frac{1}{p_i}\right)\leq l_i\leq\log_k\left(\frac{1}{p_i}\right) + 1
\]

Multiplicando todos los términos por \(p_i\):

\[
p_i\log_k\left(\frac{1}{p_i}\right)\leq p_il_i\leq p_i\log_k\left(\frac{1}{p_i}\right) + p_i
\]

y sumando ya que la desigualdad se cumple \(\forall 1\leq i\leq n\):


\begin{align*}
\sum_{i=1}^np_i\log_k\left(\frac{1}{p_i}\right)&\leq\sum_{i=1}^np_il_i\leq \sum_{i=1}^np_i\log_k\left(\frac{1}{p_i}\right) + \sum_{i=1}^np_i\\
H_k\left(FI\right)&\leq\mathcal{L}_f\leq H_k\left(FI\right) + \sum_{i=1}^np_i\\
H_k\left(FI\right)&\leq\mathcal{L}_f\leq H_k\left(FI\right) + 1\\
\end{align*}


Cumpliéndose esto también para cualquier código óptimo, lo que
constituye la formulación inicial del primer teorema de Shannon.

Puesto que esto puede aplicarse además a cualquier extensión de grado
\(m\) de una fuente de memoria nula\ldots{}


\begin{align*}
H_k\left(FI^{(m)}\right)&\leq\mathcal{L}_f^m\leq H_k\left(FI^{(m)}\right) + 1\\
mH_k\left(FI\right)&\leq\mathcal{L}_f^m\leq mH_k\left(FI\right) + 1\\
H_k\left(FI\right)&\leq\frac{\mathcal{L}_f^m}{m}\leq H_k\left(FI\right) + \frac{1}{m}\\
\end{align*}


Obteniendo así \textbf{el primer teorema de Shannon}, que es uno de los
teoremas fundamentales de la teoría de la información.

Una propiedad interesante surge cuando intentamos ver qué pasa cuando la
extensión de grado \(m\) de nuestra fuente de información de memoria
nula emite palabras largas. Para ello tenemos que ver qué sucede cuando
\(m\to\infty\):


\begin{align*}
\lim_{m\to\infty}H_k\left(FI\right)&\leq\lim_{m\to\infty}\frac{\mathcal{L}_f^m}{m}\leq \lim_{m\to\infty}H_k\left(FI\right) + \lim_{m\to\infty}\frac{1}{m}\\
H_k\left(FI\right)&\leq\lim_{m\to\infty}\frac{\mathcal{L}_f^m}{m}\leq H_k\left(FI\right) + 0\\
\end{align*}


Como \(\lim_{m\to\infty}\frac{\mathcal{L}_f^m}{m}\) queda acotado por
arriba y por abajo por \(H_k\left(FI\right)\), hemos demostrado mediante
``\emph{el teorema del sándwich}'' que

\[
\lim_{m\to\infty}\frac{\mathcal{L}_f^m}{m} = H_k\left(FI\right)
\]

Nótese que \(\frac{\mathcal{L}_f^m}{m}\) es el \textbf{número medio de
símbolos del alfabeto código (\(\mathcal{B}\)) empleados en la
codificación de un símbolo del alfabeto fuente (\(\mathcal{A}\)) cuando
se emiten secuencias de longitud \(m\), es decir, como símbolos del
alfabeto \(\mathcal{A}^{(m)}\)}.

\begin{center}\rule{0.5\linewidth}{0.5pt}\end{center}

\paragraph{Ejemplo Aplicado}\label{ejemplo-aplicado}

Sea una fuente de \(FI\) de memoria nula definida por
\(\mathcal{A}=\left\{a_1,a_2,a_3\right\}\) con
\(P=\left\{\frac{3}{4},\frac{1}{12},\frac{2}{12}\right\}\), y además
\(\mathcal{B}=\left\{0,1\right\}\). Construyámonos una tabla con
\texttt{python}, \texttt{pandas} y \texttt{numpy}:

\begin{Shaded}
\begin{Highlighting}[]
\ImportTok{import}\NormalTok{ pandas }\ImportTok{as}\NormalTok{ pd}
\ImportTok{import}\NormalTok{ numpy }\ImportTok{as}\NormalTok{ np}
\NormalTok{df }\OperatorTok{=}\NormalTok{ pd.DataFrame(}
\NormalTok{    np.array([[}\DecValTok{3}\OperatorTok{/}\DecValTok{4}\NormalTok{, }\DecValTok{1}\OperatorTok{/}\DecValTok{12}\NormalTok{, }\DecValTok{2}\OperatorTok{/}\DecValTok{12}\NormalTok{]]).T,}
\NormalTok{    columns}\OperatorTok{=}\NormalTok{[}\StringTok{"p"}\NormalTok{],}
\NormalTok{    index}\OperatorTok{=}\NormalTok{[}\StringTok{"a\_1"}\NormalTok{,}\StringTok{"a\_2"}\NormalTok{,}\StringTok{"a\_3"}\NormalTok{]}
\NormalTok{)}
\end{Highlighting}
\end{Shaded}

obteniéndose la tabla:

\begin{longtable}[]{@{}lr@{}}
\toprule\noalign{}
& p \\
\midrule\noalign{}
\endhead
\bottomrule\noalign{}
\endlastfoot
a1 & 0.75 \\
a2 & 0.0833333 \\
a3 & 0.166667 \\
\end{longtable}

\emph{(si queréis generar tablas para markdown o latex a partir de un
\texttt{DataFrame} de \texttt{pandas}, podéis emplear los métodos
\texttt{DataFrame.to\_markdown()} y \texttt{DataFrame.to\_latex()})}

Nuestro siguiente paso es calcular la información
\(I(a_i) = \log\left(\frac{1}{p_i}\right)\) para cada símbolo:

\begin{Shaded}
\begin{Highlighting}[]
\NormalTok{df[}\StringTok{"log2(1/p)"}\NormalTok{] }\OperatorTok{=}\NormalTok{ np.log2(}\DecValTok{1}\OperatorTok{/}\NormalTok{df[}\StringTok{"p"}\NormalTok{])}
\end{Highlighting}
\end{Shaded}

obteniéndose la tabla

\begin{longtable}[]{@{}lrr@{}}
\toprule\noalign{}
& p & log2(1/p) \\
\midrule\noalign{}
\endhead
\bottomrule\noalign{}
\endlastfoot
a1 & 0.75 & 0.415037 \\
a2 & 0.0833333 & 3.58496 \\
a3 & 0.166667 & 2.58496 \\
\end{longtable}

y a continuación calculamos las longitudes
\(l_i = \left\lceil\log\left(\frac{1}{p_i}\right)\right\rceil\):

\begin{Shaded}
\begin{Highlighting}[]
\NormalTok{df[}\StringTok{"l"}\NormalTok{] }\OperatorTok{=}\NormalTok{ np.ceil(df[}\StringTok{"log2(1/p)"}\NormalTok{])}
\end{Highlighting}
\end{Shaded}

obteniéndose la tabla

\begin{longtable}[]{@{}lrrr@{}}
\toprule\noalign{}
& p & log2(1/p) & l \\
\midrule\noalign{}
\endhead
\bottomrule\noalign{}
\endlastfoot
a1 & 0.75 & 0.415037 & 1 \\
a2 & 0.0833333 & 3.58496 & 4 \\
a3 & 0.166667 & 2.58496 & 3 \\
\end{longtable}

Por lo tanto podemos proponer un código instantáneo dado por estas
longitudes y empleando el alfabeto código mencionado anteriormente. Por
ejemplo:

\begin{longtable}[]{@{}lrrrr@{}}
\toprule\noalign{}
& p & log2(1/p) & l & codigo \\
\midrule\noalign{}
\endhead
\bottomrule\noalign{}
\endlastfoot
a1 & 0.75 & 0.415037 & 1 & 1 \\
a2 & 0.0833333 & 3.58496 & 4 & 0001 \\
a3 & 0.166667 & 2.58496 & 3 & 001 \\
\end{longtable}

Si quisiéramos calcular la longitud media

\begin{Shaded}
\begin{Highlighting}[]
\NormalTok{(df[}\StringTok{"p"}\NormalTok{]}\OperatorTok{*}\NormalTok{df[}\StringTok{"l"}\NormalTok{]).}\BuiltInTok{sum}\NormalTok{()}
\end{Highlighting}
\end{Shaded}

obteniéndose \(L_f\approx 1.58\). Si quisiéramos calcular la entropía

\begin{Shaded}
\begin{Highlighting}[]
\NormalTok{(df[}\StringTok{"p"}\NormalTok{]}\OperatorTok{*}\NormalTok{df[}\StringTok{"log2(1/p)"}\NormalTok{]).}\BuiltInTok{sum}\NormalTok{()}
\end{Highlighting}
\end{Shaded}

obteniéndose \(H(FI)\approx 1.04\).

\subsubsection{Rendimiento y redundancia de un código unívocamente
decodificable}\label{rendimiento-y-redundancia-de-un-cuxf3digo-unuxedvocamente-decodificable}

Se define en el contexto en el que hablábamos, el rendimiento \(\eta\)
de un código como

\[
\eta = \frac{H_k(FI)}{L_f}
\]

y la redundancia como

\[
1 - \eta = 1 - \frac{H_k(FI)}{L_f} = \frac{L_f - H_k(FI)}{L_f}
\]

Un código óptimo tiene máximo rendimiento y mínima redundancia.
